\documentclass[]{elsarticle} %review=doublespace preprint=single 5p=2 column
%%% Begin My package additions %%%%%%%%%%%%%%%%%%%
\usepackage[hyphens]{url}
\usepackage{lineno} % add
\providecommand{\tightlist}{%
  \setlength{\itemsep}{0pt}\setlength{\parskip}{0pt}}

\bibliographystyle{elsarticle-harv}
\biboptions{sort&compress} % For natbib
\usepackage{graphicx}
\usepackage{booktabs} % book-quality tables
%% Redefines the elsarticle footer
%\makeatletter
%\def\ps@pprintTitle{%
% \let\@oddhead\@empty
% \let\@evenhead\@empty
% \def\@oddfoot{\it \hfill\today}%
% \let\@evenfoot\@oddfoot}
%\makeatother

% A modified page layout
\textwidth 6.75in
\oddsidemargin -0.15in
\evensidemargin -0.15in
\textheight 9in
\topmargin -0.5in
%%%%%%%%%%%%%%%% end my additions to header

\usepackage[T1]{fontenc}
\usepackage{lmodern}
\usepackage{amssymb,amsmath}
\usepackage{ifxetex,ifluatex}
\usepackage{fixltx2e} % provides \textsubscript
% use upquote if available, for straight quotes in verbatim environments
\IfFileExists{upquote.sty}{\usepackage{upquote}}{}
\ifnum 0\ifxetex 1\fi\ifluatex 1\fi=0 % if pdftex
  \usepackage[utf8]{inputenc}
\else % if luatex or xelatex
  \usepackage{fontspec}
  \ifxetex
    \usepackage{xltxtra,xunicode}
  \fi
  \defaultfontfeatures{Mapping=tex-text,Scale=MatchLowercase}
  \newcommand{\euro}{€}
\fi
% use microtype if available
\IfFileExists{microtype.sty}{\usepackage{microtype}}{}
\ifxetex
  \usepackage[setpagesize=false, % page size defined by xetex
              unicode=false, % unicode breaks when used with xetex
              xetex]{hyperref}
\else
  \usepackage[unicode=true]{hyperref}
\fi
\hypersetup{breaklinks=true,
            bookmarks=true,
            pdfauthor={},
            pdftitle={Prediction of Consulting Project Profitability},
            colorlinks=true,
            urlcolor=blue,
            linkcolor=magenta,
            pdfborder={0 0 0}}
\urlstyle{same}  % don't use monospace font for urls
\setlength{\parindent}{0pt}
\setlength{\parskip}{6pt plus 2pt minus 1pt}
\setlength{\emergencystretch}{3em}  % prevent overfull lines
\setcounter{secnumdepth}{0}
% Pandoc toggle for numbering sections (defaults to be off)
\setcounter{secnumdepth}{0}
% Pandoc header


\usepackage[nomarkers]{endfloat}

\begin{document}
\begin{frontmatter}

  \title{Prediction of Consulting Project Profitability}
    \author[Queensland University of Technology]{Amy Cook, Paul Wu, Kerrie Mengersen\corref{c1}}
   \ead{a21.cook@qut.edu.au} 
   \cortext[c1]{Corresponding Author}
      \address[Queensland University of Technology]{School of Mathematical Sciences, George Street, Brisbane, QLD, 4000}
  
  \begin{abstract}
  A concise abstract is required. limit to 250 words. clearly state
  purpose of research, principal resutls and major conclusions. no refs
  
  Engaging in loss making jobs for fixed fees is a major problem in
  consulting, particularly in the competitive construction industry. This
  thesis investigates whether machine learning techniques applied to a
  company's passively collected internal data could help avoid loss making
  jobs or help tactfully choose when to enforce stricter contracts. It was
  found that in a specific decision framework, a case study's profits
  could be improved 9\% by declining approximately 4\% of projects.
  Alternative decision frameworks are also proposed and evaluated.
  Algorithmic methods such as Logistic Regression, Random Forests, Boosted
  Trees, Naive Bayes, and Bayesian Networks were applied as well as
  blended combinations of these methods. A decision scenario which
  rejected projects above a sequence of tested thresholds was run in order
  to find the optimal threshold for profit improvements. The blended
  Logistic Regression model outperformed other methods and produced a 95\%
  confidence interval of 6.5 - 11.5\% profit improvements. The findings
  from this research have the potential to assist managers in reducing
  losses by highlighting risky projects and guiding project-based changes
  to fee structures.
  \end{abstract}
   \begin{keyword} consulting; machine learning; profitability; predictive model;
construction industry; data mining \sep \end{keyword}
 \end{frontmatter}

\emph{Text based on elsarticle sample manuscript, see
\url{http://www.elsevier.com/author-schemas/latex-instructions\#elsarticle}}

\section{1. Introduction}\label{introduction}

Clearly state the research question and objectives of the work. Briefly
provide any necessary background to frame the research question.
Concisely summarize the major findings/results. Summary of Key Related
Research This section should include a brief summary of key related
research. Emphasis should be on demonstrating the foundation for the
current investigation. Specifically, the goal is to clearly delineate a
gap or missing link that the current research fills. Authors should
avoid presenting a litany of past research and should focus on prior
work necessary to demonstrate the existence of the research gap
addressed in the manuscript.

1 page

Intro - pick out bits from thesis

\paragraph{1.1 Problem motivation}\label{problem-motivation}

\begin{itemize}
\item
  amount of losses, sources
\item
  `outside' view, that kind of thing
\end{itemize}

\paragraph{1.2 Case Study}\label{case-study}

\section{2. Literature Review}\label{literature-review}

1 page

\paragraph{2.1 Cost estimation in the Construction Industry and IT
Industry}\label{cost-estimation-in-the-construction-industry-and-it-industry}

\paragraph{2.2 Methods used in other business
applications}\label{methods-used-in-other-business-applications}

\section{3. Methods}\label{methods}

Should provide sufficient detail to allow the work to be reproduced.
Methods already published should be indicated by a reference: only
relevant method modifications should be described.

\paragraph{3.1 Predictive methods}\label{predictive-methods}

\begin{itemize}
\tightlist
\item
  description of each of the methods: refer thesis
\end{itemize}

\paragraph{3.2 Procedure}\label{procedure}

\begin{itemize}
\tightlist
\item
  Regression, classification, blending, bottom line analysis
\end{itemize}

\section{4. Results and Discussion}\label{results-and-discussion}

Present results clearly and concisely discussion should explore the
signficance of the results of the work, not repeat them.

\paragraph{4.1 Predictive Analysis}\label{predictive-analysis}

\begin{itemize}
\item
  regression was attempted - failed
\item
  binary classification - results of 5 methods
\item
  blended models - improved results
\end{itemize}

\paragraph{4.2 Bottom line analysis}\label{bottom-line-analysis}

\begin{itemize}
\item
  decision scenario
\item
  profit curves
\end{itemize}

\section{7. Decision Support Tool}\label{decision-support-tool}

\begin{itemize}
\item
  how to generate actions from predictive results
\item
  alternative decision scenarios
\item
  user trust
\end{itemize}

\section{9. Conclusions and Future
Work}\label{conclusions-and-future-work}

3/4 page

\section{Acknowledgments}\label{acknowledgments}

This research was supported by a scholarship/ ACEMS?

\section*{References}\label{references}
\addcontentsline{toc}{section}{References}

\paragraph{Installation}\label{installation}

If the document class \emph{elsarticle} is not available on your
computer, you can download and install the system package
\emph{texlive-publishers} (Linux) or install the LaTeX~package
\emph{elsarticle} using the package manager of your TeX~installation,
which is typically TeX~Live or MikTeX.

The author names and affiliations could be formatted in two ways:

\begin{enumerate}
\def\labelenumi{(\arabic{enumi})}
\item
  Group the authors per affiliation.
\item
  Use footnotes to indicate the affiliations.
\end{enumerate}

Bullet points.

\begin{itemize}
\item
  document style
\item
  baselineskip
\item
  front matter
\item
  keywords and MSC codes
\end{itemize}

Here are two sample references: Feynman and Vernon Jr. (1963; Dirac
1953).

\hypertarget{refs}{}
\hypertarget{ref-Dirac1953888}{}
Dirac, P.A.M. 1953. ``The Lorentz Transformation and Absolute Time.''
\emph{Physica} 19 (1---12): 888--96.
doi:\href{https://doi.org/10.1016/S0031-8914(53)80099-6}{10.1016/S0031-8914(53)80099-6}.

\hypertarget{ref-Feynman1963118}{}
Feynman, R.P, and F.L Vernon Jr. 1963. ``The Theory of a General Quantum
System Interacting with a Linear Dissipative System.'' \emph{Annals of
Physics} 24: 118--73.
doi:\href{https://doi.org/10.1016/0003-4916(63)90068-X}{10.1016/0003-4916(63)90068-X}.

\end{document}


